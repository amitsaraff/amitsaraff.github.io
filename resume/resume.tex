\documentclass{res} 
\usepackage[margin=1.0in]{geometry}
\usepackage{hyperref}
\addtolength{\textheight}{.5in}
\newsectionwidth{0pt}  % So the text is not indented under section headings
\topmargin=-.4in % start text higher on the page

\begin{document}
% Center the name over the entire width of resume:
 \moveleft.5\hoffset\centerline{\large\bf Amit K. Saraff}
% Draw a horizontal line the whole width of resume:
 \moveleft\hoffset\vbox{\hrule width\resumewidth height 1pt}\smallskip
 \moveleft.5\hoffset\centerline{726 North 137th Ave, Seattle, WA 98133, USA}
 \moveleft.5\hoffset\centerline{+1 (407) 484-0146}

\begin{resume}
%\vspace{0.1in} 
\section{\centerline{Experience}} 
\vspace{2pt}

{\sl Storage Team, Mixpanel} \hfill           {\small Oct 2022 - Nov 2024} \\
{\small Seattle, WA       \hfill   (Senior Software Engineer)}
   \begin{itemize} \itemsep -2pt % reduce space between items
   \item Instrumental in driving to completion our India DC rollout to enable data residency for customers in APAC - https://mixpanel.com/blog/india-data-residency/.
   \item Tech-lead on initiative to deprecate the use of persistent disks in our storage layer. Doing so reduced our infrastructure spend, simplified the storage-tier architecture, and sped up customer queries - multiple wins!
   \item Tech-lead on initiative to optimize our infrastructure costs by single-instancing all customer historical data which resulted in reducing our overall GCP bill by \~10\%.
   \item Dev lead on a system to detect and remediate hotspots in customer data in order to reduce long tail query latencies which, in turn, impacts query runtimes and customer experience.
 \end{itemize}

{\sl xFlow, Qualtrics Corporation} \hfill           {\small Jan 2022 - Oct 2022} \\
{\small Seattle, WA       \hfill   (Staff Software Engineer)}
   \begin{itemize} \itemsep -2pt % reduce space between items
   \item Tech-lead for the xFlow platform which is Qualtrics' no/low code solution for experience management. Initiatives I 
   led and contributed to include backend transition (Amazon SWF to Temporal), data-center migrations, new site build-ups, and delayed workflow executions (new feature).
 \end{itemize}

{\sl Azure Compute, Microsoft Corporation} \hfill        {\small April 2020 - Jan 2022} \\
{\small Redmond, WA       \hfill   (Senior Software Engineer)}
   \begin{itemize} \itemsep -2pt % reduce space between items
   \item Team-lead working on reducing dock to live KPI from weeks to hours for customer managed edge zones. Reduces overall hardware lifecycle 
   costs and enables faster small footprint datacenter buildouts.
   \item Designed and implemented a zero-touch orchestration system to quickly bring up managed edge zones which ties together compute,
   storage, and slb (software load-balancer) buildouts.
%   \item Modified the Azure cloud hierarchy including subsystems and tooling to support "Managed Edge" which provides 
%    on-premise Azure capabilities delivered via management from the public cloud.
 \end{itemize}
 
{\sl OneDrive, Microsoft Corporation} \hfill        {\small March 2015 - March 2020} \\
{\small Redmond, WA       \hfill   (Senior Software Engineer)}
   \begin{itemize} \itemsep -2pt % reduce space between items
   \item Project team-lead for building an object store that applies erasure coding across datacenters for long-term storage. 
   The system provides comparable reliability and availability as the existing, globally replicated store, while reducing effective bytes stored by 33\%.
   \newline   
    Managed a team of 3 developers and released a v1 inside the OneDrive Consumer (ODC) storage stack. 
  \href{https://www.usenix.org/system/files/conference/atc17/atc17-chen_yu_lin.pdf}{Microsoft Research Paper}
   \item Built a system to manage data movement across storage tiers to optimize price performance over an object's life-cycle. Influenced through close collaboration the design 
   and development of Azure XArchive. Steady-state accomplished double-digit petabyte usage, driving down storage costs.
  \item Collaborated on a single-instance object store (in-memory, object-level deduplication with a modifiable dataset) 
   to reduce the total amount of storage transactions from high-transaction partners. At its peak, the system handled 
   up to 8\% of total daily create and update operations.
  \item Implemented an object garbage collector to detect unreferenced entries in ODC Storage. 
   This reduced over 15 petabytes of total storage in the first year, with 0 user data loss.
%  \item Extended the ODC user lookup and provisioning system to support additional services.
 \end{itemize}
 
{\sl SQL Server, Microsoft Corporation} \hfill        {\small August 2011 - March 2015} \\
{\small Redmond, WA       \hfill   (Software Engineer)}
   \begin{itemize} \itemsep -2pt % reduce space between items
   \item Tech lead for automation of release workflow process. Working with 4 developers to streamline SQL Server's
     update release process using a workflow-based system thereby reducing usage of disparate tools and vendor dependency.
   \item Worked on front-end of Query Store in Management Studio which presents a graphical
     interface to database administrators to help track top/regressed queries and allow forced plan selection.
  %   \item Developer in Full-text Search engine in Microsoft SQL Server.
%   Part of customer-centric team in charge of identifying and fixing critical and top priority requests
%   to reduce problem isolation time and support call volume.
%   \item Worked on in-house porting and integration tools to enable developers to accelerate code 
%   movement between branches. Successfully migrated between source control systems(SD to TFS) and 
%   extended application to cover distributed conflict resolution scenarios. Enabled one-click integration system
%   for cross-branch syncs.
 \end{itemize}
 
%{\sl Sensorstar, Inc.} \hfill        {\small January 2011 - July 2011} \\
%{\small Ellicott City, MD       \hfill   (Developer)}
%  \begin{itemize} \itemsep -2pt
%  \item  Architected the back-end system to store binary blobs(image, video, and audio clips) and metadata
%  with HIPAA compliance objectives for an Android app for doctors during rounds and in-home visits.
%\end{itemize} 
 
 
{\sl Physics and Astronomy, Johns Hopkins University} \hfill   {\small May 2010 - October 2010}\\
{\small Baltimore, MD       \hfill   (Summer Intern Developer)}
  \begin{itemize} \itemsep -2pt
  \item  Designed and implemented an image-processing pipeline for Hubble Space Telescope images under 
  Dr. Wei Zheng. Modified and added extensions to existing drizzle process to enable combining of multiple images from
  WCF3 and ACS cameras. Extended APSIS and multidrizzle packages to enable combining multiple arbitrary images.
\end{itemize} 

%{\sl Student Technology Services, Johns Hopkins University} \hfill  {\small April 2010 - July 2010}\\
%{\small Baltimore, MD       \hfill   (Student Developer)}
%  \begin{itemize} \itemsep -2pt
%  \item  Worked on "Virtual JHU" - a project to create an in-browser 3D representation of the Johns Hopkins
%  Homewood and Medical Campus.
%  \item Designed and implemented a partial view system to speed up initial site loads and general rendering.
%  \item  Wrote scripts to optimize/pre-render large sets of image files.
%\end{itemize} \vspace{-6pt}

%\vspace{0.1in} 
%\section{\centerline{ Skills }}
%\vspace{12pt} 
%\moveleft.5\hoffset\centerline{C\#, C, C++, Perl, Python, Git, Source Depot, TFS, GCC, WinDBG}
  
%\vspace{0.1in}
%\section{\centerline{Education}} 
%\vspace{2pt} 
%{\sl Master of Science, Computer Science} \\
%{\small Johns Hopkins University, Baltimore, MD \hspace{0.2in}  \hfill June 2011 \\}
%{\sl Bachelor of Technology, Computer Engineering}\\ % \sl will be bold italic in
%{\small Sikkim Manipal University, Majitar, Sikkim, India      \hfill  June 2007 \\}
 
\vspace{0.1in} 
\section{\centerline{Publications}} 
\vspace{12pt}
\begin{itemize}
   \item \href{https://amitsaraff.net/Zheng12_APLUS.pdf}
    {"APLUS: A Data Reduction Pipeline for HST/ACS and WFC3 Images," Astronomical Data Analysis - VII, Proceedings, May 2012}
\end{itemize}
 
 
\vspace{0.1in} 
\section{\centerline{Patents}} 
\vspace{12pt}
\begin{itemize}
   \item \href{https://patents.justia.com/patent/12020038}
    {"Peer booting operating systems on an edge network, Patent\# 12020038, Granted Jun 25, 2024}
   \item \href{https://patents.justia.com/patent/12212456}
    {"Pre-provisioning server hardware for deployment on an edge network, Patent\# 12212456, Granted Jan 28, 2025}
\end{itemize}
 
\end{resume} 
\end{document}